For each model we used state polls which only include Biden and Trump as candidates, taken between May 03, 2020 and October 31, 2020 aggregated by \cite{fivethirtyeight}.
\\~\\
We evaluate each model using a mean absolute error (MAE) over states, as shown in table \ref{tab:all_models_final_mae}, which can be interpreted as the vote share error for both candidates, since making a 3\% error in favor of Biden is equivalent to making a 3\% error against Trump. 
\begin{table}[H]
\centering
\caption{Final Results Mean Absolute Error}
\begin{tabular}{lrr}
\toprule
      model &  battleground MAE &  non-battleground MAE \\
\midrule
 ewma\_03\_lw &          0.019547 &              0.034726 \\
    ewma\_03 &          0.016324 &              {\bf 0.033381} \\
 ewma\_01\_lw &          0.023007 &              0.040461 \\
    ewma\_01 &          0.020694 &              0.036239 \\
     ewma\_1 &          {\bf 0.013867} &              0.035453 \\
\bottomrule
\end{tabular}
    \label{tab:all_models_final_mae}
\end{table}

In table \ref{tab:all_models_final_mae}, we can see that every model performs substantially better on battleground states than on non-battleground states, supporting our primary hypothesis, that battleground states will actually be more predictable than non-battleground states.
\\~\\
Interestingly, the sample size weighted models performs worse for each model compared to their unweighted versions. We hypothesized that battleground states may perform better due to more frequent polling; about twice as many polls are conducted in battleground states compared to non-battleground states, shown in Figure \ref{fig:aggregate_polling_2020-05-01_-_2020-10-31_number_of_polls}. However, despite polling more frequently, the sample size tends to be slightly smaller for battleground states in table \ref{tab:aggregate_polling_2020-05-01_-_2020-10-31_mean_sample_size}. So it may be the case that discounting polls with smaller sample sizes, might slightly negate the benefit of having more polls in battleground states. 

\begin{table}[H]
    \centering
    \begin{table}
\centering
\caption{Aggregate Polling (2020-05-01 - 2020-10-31) Mean Sample Size}
\label{table:aggregate\_polling\_2020-05-01\_-\_2020-10-31\_mean\_sample\_size}
\begin{tabular}{lr}
\toprule
 battleground &  samples mean \\
\midrule
        False &   2835.778209 \\
         True &   2659.970856 \\
\bottomrule
\end{tabular}
\end{table}

    \label{tab:aggregate_polling_2020-05-01_-_2020-10-31_mean_sample_size}
\end{table}

\subsection{Bias and Predictability}
Polling methodology varies from pollster to pollster, but demographic stratification weighting is commonly used for general election polling in the US and elsewhere, as described in \cite{Lauderdale2020ModelbasedPP}. While it may be difficult to determine whether a pollster has chosen the right post stratification weights, we can consider the effects of polls altogether to determine whether they consistently make mistakes in the same direction. Consider figure \ref{fig:ewma_01_2020-05-01-2020-10-31_signed_error}; we show the signed error for each state on the EWMA(0.1) model, showing the model has a considerable bias towards Biden for nearly every state. If we consider the mean signed error for each model in \ref{fig:ewma_03_lw_ewma_03_ewma_01_lw_ewma_01_ewma_1_mean_signed_error}, we see that every model suffers from substantial bias in favor of Biden. Since each model derives its prediction from poll data, this would suggest that pollster's methodology generally made similar mistakes in all US states. While further research is needed to identify the sources of these mistakes, since the expected value of the signed error term is substantially non-zero, a forecaster could improve predictions with a simple calibration step. With respect to predictability, this is all to say that we can expect an improvement in predictability with improved polling methodology. Since it seems that the bias affected battleground states and non-battleground states in a similar manner, these results do not necessarily suggest that battleground states are easier to predict as a result of polling methodology bias.

\begin{figure}[H]
    \centering
    \includegraphics[height=20em]{figures/ewma_01_2020-05-01-2020-10-31_signed_error.png}
    \caption{}
    \label{fig:ewma_01_2020-05-01-2020-10-31_signed_error}
\end{figure}


\begin{figure}[H]
    \centering
    \includegraphics[height=26em]{figures/ewma_03_lw_ewma_03_ewma_01_lw_ewma_01_ewma_1_mean_signed_error.png}
    \caption{}
    \label{fig:ewma_03_lw_ewma_03_ewma_01_lw_ewma_01_ewma_1_mean_signed_error}
\end{figure}


\subsection{Predictability Across Time}

Prediction results from FiveThirtyEight.com provides an interesting case study. The site publishes an updated prediction every day since June 2020 throughout right before election. \ref{fig:SignedError_538} demonstrates how far off the predictions are across different months. Each dot represents a prediction in a state. A perfect prediction would mean that all dots fall along a horizontal line at 0. The prediction error among battleground states are fall closer around 0 than that among non-battleground states.

\begin{figure}[H]
    \centering
    \includegraphics[height=26em]{figures/SignedError_538.png}
    \caption{}
    \label{fig:SignedError_538}
\end{figure}

When we use mean absolute error as the proxy of unpredictability, we also find that battleground states have lower errors. However, the subgroup means are not statistically different across all 6 months.

\begin{figure}[H]
    \centering
    \includegraphics[height=26em]{figures/SummaryMAE_538.png}
    \caption{}
    \label{fig:SummaryMAE_538}
\end{figure}

However, both figures \ref{fig:SignedError_538} and \ref{fig:SummaryMAE_538} show an unexpected result - the prediction error does not go down as the date of the prediction is closer to the actual election.

